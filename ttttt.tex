% Enable hyperlinks
\setupinteraction
  [state=start,
  style=,
  color=,
  contrastcolor=]
\setupurl[style=]

% make chapter, section bookmarks visible when opening document
\placebookmarks[chapter, section, subsection, subsubsection, subsubsubsection, subsubsubsubsection][chapter, section]
\setupinteractionscreen[option={bookmark,title}]

\setuppagenumbering[location={footer,middle}]
\setupbackend[export=yes]
\setupstructure[state=start,method=auto]

% use microtypography
\definefontfeature[default][default][script=latn, protrusion=quality, expansion=quality, itlc=yes, textitalics=yes, onum=yes, pnum=yes]
\definefontfeature[default:tnum][default][tnum=yes, pnum=no]
\definefontfeature[smallcaps][script=latn, protrusion=quality, expansion=quality, smcp=yes, onum=yes, pnum=yes]
\setupalign[hz,hanging]
\setupitaliccorrection[global, always]

\setupbodyfontenvironment[default][em=italic] % use italic as em, not slanted

\definefallbackfamily[mainface][rm][CMU Serif][preset=range:greek, force=yes]
\definefontfamily[mainface][rm][Latin Modern Roman]
\definefontfamily[mainface][mm][Latin Modern Math]
\definefontfamily[mainface][ss][Latin Modern Sans]
\definefontfamily[mainface][tt][Latin Modern Typewriter][features=none]
\setupbodyfont[mainface]

\setupwhitespace[medium]

\setuphead[chapter]            [style=\tfd\setupinterlinespace,header=empty]
\setuphead[section]            [style=\tfc\setupinterlinespace]
\setuphead[subsection]         [style=\tfb\setupinterlinespace]
\setuphead[subsubsection]      [style=\bf]
\setuphead[subsubsubsection]   [style=\sc]
\setuphead[subsubsubsubsection][style=\it]

\definesectionlevels
   [default]
   [section, subsection, subsubsection, subsubsubsection, subsubsubsubsection]

\setuphead[chapter, section, subsection, subsubsection, subsubsubsection, subsubsubsubsection][number=no]

\definedescription
  [description]
  [headstyle=bold, style=normal, location=hanging, width=broad, margin=1cm, alternative=hanging]

\setupitemize[autointro]    % prevent orphan list intro
\setupitemize[indentnext=no]

\defineitemgroup[enumerate]
\setupenumerate[each][fit][itemalign=left,distance=.5em,style={\feature[+][default:tnum]}]

\setupfloat[figure][default={here,nonumber}]
\setupfloat[table][default={here,nonumber}]

\setupxtable[frame=off]
\setupxtable[head][topframe=on]
\setupxtable[body][]
\setupxtable[foot][]
\setupxtable[lastrow][bottomframe=on]


\starttext

The executable is the new implementation of the ideas related to
PhLADiPreLiO (Phonetic Languages Approach to Discovering the Preferred
Line Options) for Ukrainian language. It uses hashes and has at the
moment (as of the version 0.11.0.0) not the full functionality. The
previous implementation and its documentation are at the links:

\goto{Old
approach}[url(https://hackage.haskell.org/package/phonetic-languages-simplified-examples-array)]

\goto{Підхід фонетичних (просодичних) мов до відкриття більш бажаних
варіантів текстового рядка (PhLADiPreLiO) з використанням
Haskell}[url(https://oleksandr-zhabenko.github.io/uk/rhythmicity/PhLADiPreLiO.Ukr.21.pdf)]

\goto{Phonetic languages approach to discovering preferred line options
(PhLADiPreLiO) using
Haskell}[url(https://oleksandrzhabenko.github.io/uk/rhythmicity/PhLADiPreLiO.Eng.21.pdf)]

The draft scientific papers on prosody and the software that are based
on the question of why some lines are easy to pronounce and read while
others are not. They are in two languages by the following links:

\goto{Чому деякі рядки легко вимовляти, а інші --- ні, або просодична
неспрогнозованість як характеристика
тексту}[url(https://www.academia.edu/105067723/\%D0\%A7\%D0\%BE\%D0\%BC\%D1\%83_\%D0\%B4\%D0\%B5\%D1\%8F\%D0\%BA\%D1\%96_\%D1\%80\%D1\%8F\%D0\%B4\%D0\%BA\%D0\%B8_\%D0\%BB\%D0\%B5\%D0\%B3\%D0\%BA\%D0\%BE_\%D0\%B2\%D0\%B8\%D0\%BC\%D0\%BE\%D0\%B2\%D0\%BB\%D1\%8F\%D1\%82\%D0\%B8_\%D0\%B0_\%D1\%96\%D0\%BD\%D1\%88\%D1\%96_\%D0\%BD\%D1\%96_\%D0\%B0\%D0\%B1\%D0\%BE_\%D0\%BF\%D1\%80\%D0\%BE\%D1\%81\%D0\%BE\%D0\%B4\%D0\%B8\%D1\%87\%D0\%BD\%D0\%B0_\%D0\%BD\%D0\%B5\%D1\%81\%D0\%BF\%D1\%80\%D0\%BE\%D0\%B3\%D0\%BD\%D0\%BE\%D0\%B7\%D0\%BE\%D0\%B2\%D0\%B0\%D0\%BD\%D1\%96\%D1\%81\%D1\%82\%D1\%8C_\%D1\%8F\%D0\%BA_\%D1\%85\%D0\%B0\%D1\%80\%D0\%B0\%D0\%BA\%D1\%82\%D0\%B5\%D1\%80\%D0\%B8\%D1\%81\%D1\%82\%D0\%B8\%D0\%BA\%D0\%B0_\%D1\%82\%D0\%B5\%D0\%BA\%D1\%81\%D1\%82\%D1\%83)]

\goto{Why some lines are easy to pronounce and others are not, or
prosodic unpredictability as a characteristic of
text}[url(https://www.academia.edu/105067761/Why_some_lines_are_easy_to_pronounce_and_others_are_not_or_prosodic_unpredictability_as_a_characteristic_of_text)]

The old documentation for the implementation is available at the
following links. \goto{Більш бажані варіанти текстового рядка на основі
PhLADiPreLiO з використанням Haskell --- базові
ідеї}[url(https://oleksandr-zhabenko.github.io/uk/rhythmicity/phladiprelioUkr.7.pdf)]

\goto{Preferred line options based on the PhLADiPreLiO using
Haskell}[url(https://oleksandr-zhabenko.github.io/uk/rhythmicity/phladiprelioEng.7.pdf)]

The examples of the using the new functionality in the 0.10.0.0 version
is in two short videos by the links below:

\goto{+f mode
additions}[url(https://www.facebook.com/Oleksandr.S.Zhabenko/videos/593118766229253/)]

\goto{+m mode --- new one in this
version}[url(https://www.facebook.com/Oleksandr.S.Zhabenko/videos/325188039864718/)]

Since the version 0.7.0.0 the default mode for non-tests output is in
two-column. This improves the UI / UX for the end user. Besides there is
the possibility to write the result to a file.

The video with the demonstration of the new functionality in the version
0.7.0.0 is by the link:

\goto{new functionality in
0.7.0.0}[url(https://www.facebook.com/100012184148486/posts/1767318773684244)]

Three videos with examples of the usage of the new music mode are by the
following link:
https://www.facebook.com/Oleksandr.S.Zhabenko/posts/pfbid02Vtcxuo5d73ZqsmgbxoRxLJoxLLmfpZ5B4VB9g7AQzuVTnydLHVtGRD48Q8RWLy2dl

Examples of the new functionality in the version 0.5.1.1 are in the
videos:

https://www.facebook.com/Oleksandr.S.Zhabenko/posts/pfbid033gzq8MCRQsm65mPrzJL25MZNgvW7mezQSywULiVMnqmTBMtSW2jW4ABh6HVMWZNLl

https://www.facebook.com/Oleksandr.S.Zhabenko/posts/pfbid05mMCUu5HVoA6aV4kLJ69tQZHyhTRZgXLRvtdLitWCm6JeB2T2ktfkd2opfjjgTxFl

Video recordings as examples of the working prototype usage and how you
can listen to Ukrainian text using Google services are at the links
below:

https://www.facebook.com/Oleksandr.S.Zhabenko/videos/796964592047546

https://www.facebook.com/Oleksandr.S.Zhabenko/posts/pfbid02EaC4Zwn4YVjfVFWpUhLoHYomiFHZQaiMdorLa6PPx9kXBXepTYPFEFMv8iyV4wAYl

See the demonstration video for the new functionality in version
0.12.0.0 and 0.12.0.1 by the link:
\useURL[url1][https://www.youtube.com/watch?v=zapAPpQ-fc4]\from[url1]

Since the version 0.13.0.0 there is a possibility to compare the
distances (opposite to the similarity measure) for the line options
using the +l2 \ldots{} -l2 command line group of options. It is
generally a completely new functionality for the package.

Since the version 0.14.0.0 there is a possibility to change the
durations of the selected syllables using the =\{set of digits\}
precisely after the needed syllable. For more information, see the
output of the call of the program with the -h command line argument.
This significantly extends the general possibilities of the program,
especially for the music composing.

Since the version 0.15.0.0 there were fixed issues with distance between
line options in several branches and added a possibility to analyse and
compare two lines from the same file using either additionally to +m
also +m2 group of command line arguments, or +m3 group of arguments
instead.

To see the help message with synopsis information, run:

phladiprelioUkr -h

Since the version 0.15.1.0 there is a possibility to see selective help
just for one command line option (e. g. +dc) with general context or
without it. To get the help message for the option with general context
(extended help for one command line argument), use combination of
\quotation{-h } (e. g. -h +dc). To get just help on this option without
context, use the option without \quotation{+} or \quotation{-} (e. g.
\quotation{-h dc}).

The version 0.15.3.0 adds some performance improvements and fixes the
issue with just one syllable before \quote{=} symbol in music mode.

For the list of bash aliases a few of which are used in the videos, see:
https://github.com/Oleksandr-Zhabenko/phladiprelio-alias/blob/main/.bashrc

Breaking changes since 0.20.0.0 ===============================

Since the version 0.20.0.0 the way of calculation is significantly
changed, therefore the results are expected to be different. Please,
have this in mind. It is expected to have larger groups and more options
to choose from. Nevertheless, the approach becomes more stable. The
performance and resources consumtion are very closely comparable with
probably some subtle differences between versions 0.15.* and 0.20.0.0.

Devotion ========

P.S.: the author would like to devote this project to support the
Foundation GASTROSTARS. The version 0.9.0.0 is devoted to Vico Kok.

At the day of publication of the first version of the package
(12/03/2023) there is the foundation founder's (this is \goto{Emma
Kok}[url(https://www.emmakok.nl)]) Birthday.

And on the 17/03/2023 there is the author's Birthday.

And on the 19/03/2023 is the St.~Joseph the Betrothed feast for Roman
Catholics.

On the 19/04/2023 there is Emma's namesday, the memory of St.~Emma of
Lesum or Emma of Stiepel (also known as Hemma and Imma) Day.

On the 14/05/2023 there is Mother's Day. It is also a good opportunity
to support the foundation.

On the 25/05/2023 there is Ascension Day according to the Julian Greek
Orthodox calendar.

On the 01/06/2023 many countries including Ukraine and the Netherlands
celebrate Children's Day.

On the 09/06/2023 there was unofficial World Friendship Day.

On the 15/06/2023 there is the final bachelor's exam for Enzo Kok in
violin playing in Amsterdam.

On the 30/07/2023 there is the Birthday of Vico Kok and also
International Friendship Day. Therefore, the release versions 0.9.0.0
and 0.9.0.1 are devoted to Vico.

The version 0.10.0.0 is devoted to \goto{Enzo
Kok}[url(https://enzokok.nl)] whose Birthday is on the 14th of August.

On the 01/10/2023 there are World Music Day and André's Rieu Birthday.
So the versions 0.11.0.0 and 0.11.0.1 are additionally devoted to him as
well as to the Kok family --- Emma and Enzo are amazing and fantastic
musicians, Vico works with PhilZuid, Sophie sings in the classical
manner. Nathalie Kok with love to her family and appreciation for
André's Rieu support also celebrates the Day.

In Ukraine this day has several important celebrations --- Intercession
of the Holy Theotokos (Pokrova), Cossacks' Day, Day of Defenders of
Ukraine (especially relevant during the Russian war against Ukraine).

On the 27/10/2023 there is Ukrainian Language and Writing Day.

On the 11/11/2023 there is St.~Martin of Tours Day for Roman Catholics
and Poland Independence Day.

The version 0.15.0.0 is also devoted to the bright memory of the Artem
Sachuk, who perished as a soldier defending Ukraine from the Russian
occupants. Kingdom of God to his soul and eternal memory! Condolences to
everybody who knows him.

The version 0.15.3.0 is also devoted to Sophie Kok, a sister of Emma
Kok, who turned 18 on the 06/01/2024. Besides, on the 22/01/2024 there
is Day of Unity of Ukraine and on the 23/01/2024 there is for Orthodox
Church memory of St.~Paulinus of Nola.

On the 08/03/2024 there is International Women's Day.

On the 12/03/2024 there was a Birthday of Emma Kok, the foundation
founder. She turned 16. Therefore, the release 0.20.0.0 is additionally
devoted i.e.~tributed to her.

Besides, you can support Ukrainian people in various forms.

All support is welcome, including donations for the needs of the
Ukrainian army, IDPs and refugees.

If you would like to share some financial support, please, contact the
mentioned foundation using the URL:

\goto{Contact Foundation
GASTROSTARS}[url(https://gastrostars.nl/hou-mij-op-de-hoogte)]

or

\goto{Donation Page}[url(https://gastrostars.nl/doneren)]

\stoptext
